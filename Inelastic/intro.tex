\section{\label{sec:level1} Introduction}
Astrophysical and cosmological evidence indicates that the dominant mass fraction of our Universe consists of some yet unknown
form of dark, or invisible matter. The dark matter could be made of new, yet undiscovered particles, and well-motivated
theoretical models going beyond the Standard Model of particle physics predict the existence of Weakly Interacting Massive
Particles (WIMPs), which are natural candidates for dark matter. This hypothesis is currently being tested by several direct
and indirect detection experiments, as well as at the LHC~\cite{Bertone:2004pz,Baudis:2016qwx}.

Most direct detection searches focus on elastic scattering of galactic dark matter particles off nuclei, where the keV-scale 
nuclear recoil energy is to be detected~\cite{Baudis:2012ig,Baudis:2015mpa,Undagoitia:2015gya}. In this work, the 
alternative process of inelastic scattering is explored, where a WIMP-nucleus scattering induces a transition to a low-lying 
excited nuclear state. The experimental signature is a nuclear recoil detected together with the prompt de-excitation 
photon~\cite{Ellis:1988nb}. 

We consider the $^{129}\text{Xe}$ isotope, which has an abundance of 26.4\% in natural xenon, and a lowest-lying 
3/2$^{+}$ state at 36.6\,keV above the 1/2$^+$ gound state. The electromnagnetic nuclear decay has a half-life of 0.97\,s. 
The sigantures of inelastic scattering in xenon have been studied in detail in~\cite{Baudis:2013bba}. It was found that this 
channel is complementary to spin-dependent, elastic scattering, dominating the integrated rates above  $\simeq10$\,keV energy 
depositions. In addition, in case of a positive signal, the observation of inelastic scattering would provide a clear 
indication of the spin-dependent nature of the fundamental interaction. 

\section{Xenon100 Detector}
The Xenon100 experiment is a  dual phase liquid xenon TPC. For a given interaction in the liquid target this type of detector produces two separated signals,
one proportional to the prompt scintillation (S1) the other to ionization (S2).

\textcolor{blue}{To add: describe and explain cS1 and cS2 definitions, some sentences about detector stability, maybe science run data used and calibration campaign goes here, 
maybe Ly and Y measurements used goes here.} 

